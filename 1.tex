% Options for packages loaded elsewhere
\PassOptionsToPackage{unicode}{hyperref}
\PassOptionsToPackage{hyphens}{url}
%
\documentclass[
]{article}
\usepackage{amsmath,amssymb}
\usepackage{iftex}
\ifPDFTeX
  \usepackage[T1]{fontenc}
  \usepackage[utf8]{inputenc}
  \usepackage{textcomp} % provide euro and other symbols
\else % if luatex or xetex
  \usepackage{unicode-math} % this also loads fontspec
  \defaultfontfeatures{Scale=MatchLowercase}
  \defaultfontfeatures[\rmfamily]{Ligatures=TeX,Scale=1}
\fi
\usepackage{lmodern}
\ifPDFTeX\else
  % xetex/luatex font selection
\fi
% Use upquote if available, for straight quotes in verbatim environments
\IfFileExists{upquote.sty}{\usepackage{upquote}}{}
\IfFileExists{microtype.sty}{% use microtype if available
  \usepackage[]{microtype}
  \UseMicrotypeSet[protrusion]{basicmath} % disable protrusion for tt fonts
}{}
\makeatletter
\@ifundefined{KOMAClassName}{% if non-KOMA class
  \IfFileExists{parskip.sty}{%
    \usepackage{parskip}
  }{% else
    \setlength{\parindent}{0pt}
    \setlength{\parskip}{6pt plus 2pt minus 1pt}}
}{% if KOMA class
  \KOMAoptions{parskip=half}}
\makeatother
\usepackage{xcolor}
\usepackage[margin=1in]{geometry}
\usepackage{color}
\usepackage{fancyvrb}
\newcommand{\VerbBar}{|}
\newcommand{\VERB}{\Verb[commandchars=\\\{\}]}
\DefineVerbatimEnvironment{Highlighting}{Verbatim}{commandchars=\\\{\}}
% Add ',fontsize=\small' for more characters per line
\usepackage{framed}
\definecolor{shadecolor}{RGB}{248,248,248}
\newenvironment{Shaded}{\begin{snugshade}}{\end{snugshade}}
\newcommand{\AlertTok}[1]{\textcolor[rgb]{0.94,0.16,0.16}{#1}}
\newcommand{\AnnotationTok}[1]{\textcolor[rgb]{0.56,0.35,0.01}{\textbf{\textit{#1}}}}
\newcommand{\AttributeTok}[1]{\textcolor[rgb]{0.13,0.29,0.53}{#1}}
\newcommand{\BaseNTok}[1]{\textcolor[rgb]{0.00,0.00,0.81}{#1}}
\newcommand{\BuiltInTok}[1]{#1}
\newcommand{\CharTok}[1]{\textcolor[rgb]{0.31,0.60,0.02}{#1}}
\newcommand{\CommentTok}[1]{\textcolor[rgb]{0.56,0.35,0.01}{\textit{#1}}}
\newcommand{\CommentVarTok}[1]{\textcolor[rgb]{0.56,0.35,0.01}{\textbf{\textit{#1}}}}
\newcommand{\ConstantTok}[1]{\textcolor[rgb]{0.56,0.35,0.01}{#1}}
\newcommand{\ControlFlowTok}[1]{\textcolor[rgb]{0.13,0.29,0.53}{\textbf{#1}}}
\newcommand{\DataTypeTok}[1]{\textcolor[rgb]{0.13,0.29,0.53}{#1}}
\newcommand{\DecValTok}[1]{\textcolor[rgb]{0.00,0.00,0.81}{#1}}
\newcommand{\DocumentationTok}[1]{\textcolor[rgb]{0.56,0.35,0.01}{\textbf{\textit{#1}}}}
\newcommand{\ErrorTok}[1]{\textcolor[rgb]{0.64,0.00,0.00}{\textbf{#1}}}
\newcommand{\ExtensionTok}[1]{#1}
\newcommand{\FloatTok}[1]{\textcolor[rgb]{0.00,0.00,0.81}{#1}}
\newcommand{\FunctionTok}[1]{\textcolor[rgb]{0.13,0.29,0.53}{\textbf{#1}}}
\newcommand{\ImportTok}[1]{#1}
\newcommand{\InformationTok}[1]{\textcolor[rgb]{0.56,0.35,0.01}{\textbf{\textit{#1}}}}
\newcommand{\KeywordTok}[1]{\textcolor[rgb]{0.13,0.29,0.53}{\textbf{#1}}}
\newcommand{\NormalTok}[1]{#1}
\newcommand{\OperatorTok}[1]{\textcolor[rgb]{0.81,0.36,0.00}{\textbf{#1}}}
\newcommand{\OtherTok}[1]{\textcolor[rgb]{0.56,0.35,0.01}{#1}}
\newcommand{\PreprocessorTok}[1]{\textcolor[rgb]{0.56,0.35,0.01}{\textit{#1}}}
\newcommand{\RegionMarkerTok}[1]{#1}
\newcommand{\SpecialCharTok}[1]{\textcolor[rgb]{0.81,0.36,0.00}{\textbf{#1}}}
\newcommand{\SpecialStringTok}[1]{\textcolor[rgb]{0.31,0.60,0.02}{#1}}
\newcommand{\StringTok}[1]{\textcolor[rgb]{0.31,0.60,0.02}{#1}}
\newcommand{\VariableTok}[1]{\textcolor[rgb]{0.00,0.00,0.00}{#1}}
\newcommand{\VerbatimStringTok}[1]{\textcolor[rgb]{0.31,0.60,0.02}{#1}}
\newcommand{\WarningTok}[1]{\textcolor[rgb]{0.56,0.35,0.01}{\textbf{\textit{#1}}}}
\usepackage{graphicx}
\makeatletter
\def\maxwidth{\ifdim\Gin@nat@width>\linewidth\linewidth\else\Gin@nat@width\fi}
\def\maxheight{\ifdim\Gin@nat@height>\textheight\textheight\else\Gin@nat@height\fi}
\makeatother
% Scale images if necessary, so that they will not overflow the page
% margins by default, and it is still possible to overwrite the defaults
% using explicit options in \includegraphics[width, height, ...]{}
\setkeys{Gin}{width=\maxwidth,height=\maxheight,keepaspectratio}
% Set default figure placement to htbp
\makeatletter
\def\fps@figure{htbp}
\makeatother
\setlength{\emergencystretch}{3em} % prevent overfull lines
\providecommand{\tightlist}{%
  \setlength{\itemsep}{0pt}\setlength{\parskip}{0pt}}
\setcounter{secnumdepth}{-\maxdimen} % remove section numbering
\ifLuaTeX
  \usepackage{selnolig}  % disable illegal ligatures
\fi
\usepackage{bookmark}
\IfFileExists{xurl.sty}{\usepackage{xurl}}{} % add URL line breaks if available
\urlstyle{same}
\hypersetup{
  pdftitle={HW1},
  hidelinks,
  pdfcreator={LaTeX via pandoc}}

\title{HW1}
\author{}
\date{\vspace{-2.5em}2024-10-22}

\begin{document}
\maketitle

\begin{Shaded}
\begin{Highlighting}[]
\FunctionTok{load}\NormalTok{(}\StringTok{"ants.Rdata"}\NormalTok{)}
\FunctionTok{summary}\NormalTok{(ants}\SpecialCharTok{$}\NormalTok{abundance)}
\end{Highlighting}
\end{Shaded}

\begin{verbatim}
##    Min. 1st Qu.  Median    Mean 3rd Qu.    Max. 
##    0.00    6.75   35.50   86.31   99.25  767.00
\end{verbatim}

\begin{Shaded}
\begin{Highlighting}[]
\FunctionTok{summary}\NormalTok{(ants}\SpecialCharTok{$}\NormalTok{moisture)}
\end{Highlighting}
\end{Shaded}

\begin{verbatim}
##    Min. 1st Qu.  Median    Mean 3rd Qu.    Max. 
##   12.00   15.40   17.95   17.52   19.90   21.50
\end{verbatim}

\begin{Shaded}
\begin{Highlighting}[]
\FunctionTok{var}\NormalTok{(ants}\SpecialCharTok{$}\NormalTok{abundance)}
\end{Highlighting}
\end{Shaded}

\begin{verbatim}
## [1] 19882.49
\end{verbatim}

\begin{Shaded}
\begin{Highlighting}[]
\FunctionTok{var}\NormalTok{(ants}\SpecialCharTok{$}\NormalTok{moisture)}
\end{Highlighting}
\end{Shaded}

\begin{verbatim}
## [1] 7.083322
\end{verbatim}

\begin{Shaded}
\begin{Highlighting}[]
\FunctionTok{hist}\NormalTok{(ants}\SpecialCharTok{$}\NormalTok{abundance, }\AttributeTok{main =} \StringTok{"Histogram of Ant Abundance"}\NormalTok{, }\AttributeTok{xlab =} \StringTok{"Abundance"}\NormalTok{, }\AttributeTok{ylab =} \StringTok{"Frequency"}\NormalTok{)}
\end{Highlighting}
\end{Shaded}

\includegraphics{1_files/figure-latex/task1-1.pdf}

\begin{Shaded}
\begin{Highlighting}[]
\FunctionTok{boxplot}\NormalTok{(ants}\SpecialCharTok{$}\NormalTok{abundance, }\AttributeTok{main =} \StringTok{"Boxplot of Ant Abundance"}\NormalTok{, }\AttributeTok{ylab =} \StringTok{"Abundance"}\NormalTok{)}
\end{Highlighting}
\end{Shaded}

\includegraphics{1_files/figure-latex/task1-2.pdf}

\begin{Shaded}
\begin{Highlighting}[]
\FunctionTok{plot}\NormalTok{(ants}\SpecialCharTok{$}\NormalTok{moisture, ants}\SpecialCharTok{$}\NormalTok{abundance, }
     \AttributeTok{main =} \StringTok{"Scatter Plot of Moisture vs Ant Abundance"}\NormalTok{, }
     \AttributeTok{xlab =} \StringTok{"Moisture (\%)"}\NormalTok{, }\AttributeTok{ylab =} \StringTok{"Ant Abundance"}\NormalTok{)}
\end{Highlighting}
\end{Shaded}

\includegraphics{1_files/figure-latex/task1-3.pdf}

To define the likelihood function in R, we use a function that computes
the product of probabilities for all observations.

\begin{Shaded}
\begin{Highlighting}[]
\CommentTok{\# Define the likelihood function}
\NormalTok{likelihood\_function }\OtherTok{\textless{}{-}} \ControlFlowTok{function}\NormalTok{(params, abundance, moisture) \{}
\NormalTok{  beta\_0 }\OtherTok{\textless{}{-}}\NormalTok{ params[}\DecValTok{1}\NormalTok{]  }\CommentTok{\# The intercept parameter}
\NormalTok{  beta\_1 }\OtherTok{\textless{}{-}}\NormalTok{ params[}\DecValTok{2}\NormalTok{]  }\CommentTok{\# The coefficient for moisture}
\NormalTok{  phi }\OtherTok{\textless{}{-}}\NormalTok{ params[}\DecValTok{3}\NormalTok{]     }\CommentTok{\# The overdispersion parameter}
  
  \CommentTok{\# Calculate the mean (mu) for each observation}
\NormalTok{  mu }\OtherTok{\textless{}{-}} \FunctionTok{exp}\NormalTok{(beta\_0 }\SpecialCharTok{+}\NormalTok{ beta\_1 }\SpecialCharTok{*}\NormalTok{ moisture)}
  
  \CommentTok{\# Compute the negative binomial likelihood for each observation}
\NormalTok{  likelihoods }\OtherTok{\textless{}{-}} \FunctionTok{dnbinom}\NormalTok{(abundance, }\AttributeTok{size =} \DecValTok{1}\SpecialCharTok{/}\NormalTok{phi, }\AttributeTok{mu =}\NormalTok{ mu, }\AttributeTok{log =} \ConstantTok{FALSE}\NormalTok{)}
  
  \CommentTok{\# Calculate the total likelihood by multiplying individual likelihoods}
\NormalTok{  total\_likelihood }\OtherTok{\textless{}{-}} \FunctionTok{prod}\NormalTok{(likelihoods)}
  
  \FunctionTok{return}\NormalTok{(total\_likelihood)}
\NormalTok{\}}

\NormalTok{initial\_params }\OtherTok{\textless{}{-}} \FunctionTok{c}\NormalTok{(}\DecValTok{1}\NormalTok{, }\FloatTok{0.1}\NormalTok{, }\FloatTok{0.5}\NormalTok{)  }\CommentTok{\# Example starting values for beta\_0, beta\_1, and phi}
\NormalTok{likelihood\_value }\OtherTok{\textless{}{-}} \FunctionTok{likelihood\_function}\NormalTok{(initial\_params, ants}\SpecialCharTok{$}\NormalTok{abundance, ants}\SpecialCharTok{$}\NormalTok{moisture)}
\NormalTok{likelihood\_value}
\end{Highlighting}
\end{Shaded}

\begin{verbatim}
## [1] 0
\end{verbatim}

If we use the set of parameters described in the example above, the
likelihood value results 0. One possible reason is that the number is
too small to exhibit. That's why we derive the following log likelihood
function.

\begin{Shaded}
\begin{Highlighting}[]
\CommentTok{\# Define the log{-}likelihood function}
\NormalTok{log\_likelihood }\OtherTok{\textless{}{-}} \ControlFlowTok{function}\NormalTok{(params, abundance, moisture) \{}
\NormalTok{  beta\_0 }\OtherTok{\textless{}{-}}\NormalTok{ params[}\DecValTok{1}\NormalTok{]  }\CommentTok{\# The intercept parameter}
\NormalTok{  beta\_1 }\OtherTok{\textless{}{-}}\NormalTok{ params[}\DecValTok{2}\NormalTok{]  }\CommentTok{\# The coefficient for moisture}
\NormalTok{  phi }\OtherTok{\textless{}{-}}\NormalTok{ params[}\DecValTok{3}\NormalTok{]     }\CommentTok{\# The overdispersion parameter}
  
  \CommentTok{\# Calculate the mean (mu) for each observation}
\NormalTok{  mu }\OtherTok{\textless{}{-}} \FunctionTok{exp}\NormalTok{(beta\_0 }\SpecialCharTok{+}\NormalTok{ beta\_1 }\SpecialCharTok{*}\NormalTok{ moisture)}
  
  \CommentTok{\# Compute the log{-}likelihood for each observation}
\NormalTok{  log\_lik }\OtherTok{\textless{}{-}} \FunctionTok{sum}\NormalTok{(}
    \FunctionTok{lgamma}\NormalTok{(abundance }\SpecialCharTok{+} \DecValTok{1} \SpecialCharTok{/}\NormalTok{ phi) }\SpecialCharTok{{-}} \FunctionTok{lgamma}\NormalTok{(}\DecValTok{1} \SpecialCharTok{/}\NormalTok{ phi) }\SpecialCharTok{{-}} \FunctionTok{lgamma}\NormalTok{(abundance }\SpecialCharTok{+} \DecValTok{1}\NormalTok{) }\SpecialCharTok{+}
\NormalTok{    (}\DecValTok{1} \SpecialCharTok{/}\NormalTok{ phi) }\SpecialCharTok{*} \FunctionTok{log}\NormalTok{(}\DecValTok{1} \SpecialCharTok{/}\NormalTok{ (}\DecValTok{1} \SpecialCharTok{+}\NormalTok{ mu }\SpecialCharTok{*}\NormalTok{ phi)) }\SpecialCharTok{+}
\NormalTok{    abundance }\SpecialCharTok{*} \FunctionTok{log}\NormalTok{(mu }\SpecialCharTok{*}\NormalTok{ phi }\SpecialCharTok{/}\NormalTok{ (}\DecValTok{1} \SpecialCharTok{+}\NormalTok{ mu }\SpecialCharTok{*}\NormalTok{ phi))}
\NormalTok{  )}
  
  \FunctionTok{return}\NormalTok{(log\_lik)  }\CommentTok{\# Return the log{-}likelihood}
\NormalTok{\}}

\NormalTok{initial\_params }\OtherTok{\textless{}{-}} \FunctionTok{c}\NormalTok{(}\DecValTok{1}\NormalTok{, }\FloatTok{0.1}\NormalTok{, }\FloatTok{0.5}\NormalTok{)  }\CommentTok{\# Example starting values for beta\_0, beta\_1, and phi}
\NormalTok{log\_likelihood\_value }\OtherTok{\textless{}{-}} \FunctionTok{log\_likelihood}\NormalTok{(initial\_params, ants}\SpecialCharTok{$}\NormalTok{abundance, ants}\SpecialCharTok{$}\NormalTok{moisture)}
\NormalTok{log\_likelihood\_value}
\end{Highlighting}
\end{Shaded}

\begin{verbatim}
## [1] -1148.303
\end{verbatim}

Using the same set of parameters, the log likelihood values result
-1148, which makes more sense.

\end{document}
